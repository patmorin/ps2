\documentclass{patmorin}
\listfiles
\usepackage{pat}
\usepackage{paralist}
\usepackage{dsfont}  % for \mathds{A}
\usepackage[utf8x]{inputenc}
\usepackage{skull}
\usepackage{paralist}
\usepackage{graphicx}
\usepackage[noend]{algorithmic}

\usepackage[normalem]{ulem}
\usepackage{cancel}
\usepackage{enumitem}

\usepackage{todonotes}

\usepackage[longnamesfirst,numbers,sort&compress]{natbib}

\usepackage[mathlines]{lineno}
\setlength{\linenumbersep}{2em}
% \linenumbers
% \rightlinenumbers
% \linenumbers
\newcommand*\patchAmsMathEnvironmentForLineno[1]{%
 \expandafter\let\csname old#1\expandafter\endcsname\csname #1\endcsname
 \expandafter\let\csname oldend#1\expandafter\endcsname\csname end#1\endcsname
 \renewenvironment{#1}%
    {\linenomath\csname old#1\endcsname}%
    {\csname oldend#1\endcsname\endlinenomath}}%
\newcommand*\patchBothAmsMathEnvironmentsForLineno[1]{%
 \patchAmsMathEnvironmentForLineno{#1}%
 \patchAmsMathEnvironmentForLineno{#1*}}%
\AtBeginDocument{%
\patchBothAmsMathEnvironmentsForLineno{equation}%
\patchBothAmsMathEnvironmentsForLineno{align}%
\patchBothAmsMathEnvironmentsForLineno{flalign}%
\patchBothAmsMathEnvironmentsForLineno{alignat}%
\patchBothAmsMathEnvironmentsForLineno{gather}%
\patchBothAmsMathEnvironmentsForLineno{multline}%
}


\newcommand{\coloured}[2]{{\color{#1}{#2}}}
\newenvironment{colourblock}[1]{\color{#1}}{}

\newcommand{\condref}[1]{(C\ref{#1})}

% Taken from
% https://tex.stackexchange.com/questions/42726/align-but-show-one-equation-number-at-the-end
\newcommand\numberthis{\addtocounter{equation}{1}\tag{\theequation}}


\setlength{\parskip}{1ex}


\DeclareMathOperator{\diam}{diam}
\DeclareMathOperator{\tw}{tw}
\DeclareMathOperator{\stw}{stw}
\DeclareMathOperator{\ltw}{ltw}
\DeclareMathOperator{\pw}{pw}
\DeclareMathOperator{\lpw}{lpw}
\DeclareMathOperator{\lhptw}{lhp-tw}
\DeclareMathOperator{\lhppw}{lhp-pw}

\DeclareMathOperator{\x}{x}
\DeclareMathOperator{\depth}{d}
\DeclareMathOperator{\sh}{cbt}
\DeclareMathOperator{\cbt}{cbt}
\DeclareMathOperator{\sgn}{sgn}
\DeclareMathOperator{\dc}{dc}

\DeclareMathOperator{\afci}{\overline{\chi}_\pi}
\DeclareMathOperator{\afcn}{\dot{\chi}_\pi}

\newcommand{\ellt}{{\lfloor\ell/2\rfloor}}

\title{\MakeUppercase{A New Proof of the Planar Product Structure Theorem}\thanks{This research was partly funded by NSERC.}}
\author{Prosenjit Bose and Pat Morin%
    \thanks{School of Computer Science, Carleton University}}

\date{}

\DeclareMathOperator{\ddiv}{div}
\DeclareMathOperator{\hist}{p}

\newcommand{\colored}[2]{{\color{#1}#2}}

\usepackage{tabularx}

\DeclareMathOperator{\ci}{\overline{\pi}}

\begin{document}

% \begin{titlepage}
\maketitle

\begin{abstract}
    The Product Structure Theorem for planar graphs (Dujmović et al, 2019) states that any planar graph is contained in strong product of a planar $3$-tree, a path, and $3$-cycle.  We give an alternative proof of this theorem.
\end{abstract}
% \end{titlepage}

% \pagenumbering{roman}
% \tableofcontents
%
% \newpage
% \pagenumbering{arabic}

\section{Introduction}

\section{Tripod Decomposition}


Let $G$ be a $n$-vertex triangulation and $T_0$ be a breadth-first spanning forest\todo{Define BFS forest} of $G$ having three roots, all on a common face, $f_0$, of $G$.  A \emph{tripod} $\tau$ in $G$ is a triple of disjoint vertical paths\todo{Define vertical paths and lower endpoints} in $T_0$ whose lower endpoints are contained in a single face of $G$.  (Note that each of the three paths in $\tau$ may be empty.)

Starting in the next sentence, we will alternately treat tripods as triples of paths and as vertex sets, relying on the reader to distinguish between the two from context.  A \emph{tripod decomposition} $\mathcal{D}$ of $(G,T_0)$ is a partition of $V(G)$ into tripods.  \citet{dujmovic.joret.ea:planar} proved the following result, which implies the planar product structure theorem:

\begin{thm}\label{tripod_decomposition}
  For any triangulation $G$ and any BFS-tree $T_0$ of $G$, there exists a tripod decomposition $\mathcal{D}$ of $(G,T_0)$ such that $\tw(G/\mathcal{D})\le 3$.
\end{thm}

Let $\mathcal{F}:=f_0,\ldots,f_{2n-3}$ be a total ordering of the faces of $G$.  Then $\mathcal{F}$ defines a tripod decomposition $\mathcal{D}_\mathcal{F}:=\{\tau_0,\ldots,\tau_{2n-3}\}$ of $(G,T_0)$ as follows:
\begin{compactenum}
  \item $\tau_0$ is the tripod consisting of the three roots of $T_0$.
  \item For each $i\in\{0,\ldots,2n-2\}$, let $S_i:=\bigcup_{j=0}^i \tau_i$.
  \item For each $i\in\{1,\ldots,2n-3\}$, let $P_i$, $Q_i$, and $R_i$ be the three paths in $T_0$ from the vertices of $f_i$ to the roots of their respect trees in $T_0$.  The tripod $\tau_i$ consists of $P_i-S_{i-1}$, $Q_i-S_{i-1}$ and $R_i-S_{i-1}$.
\end{compactenum}
It is straightforward to verify from these definitions that $\mathcal{D}_\mathcal{F}$ is indeed a tripod decomposition of $(G,T_0)$.

Let $\overline{\tau}_0$ be the subgraph of $G$ consisting of the edges and vertices of $f_0$.  For each $i\in\{0,\ldots,2n-3\}$, let $\overline{\tau}_i$ denote the subgraph of $G$ consisting of the edges and vertices of $f_i$, the edges and vertices of the three paths in $\tau_i$ and, for each non-empty path in $\tau$ with top vertex $v$, the edge from $v$ to its $T_0$-parent.  We let $G_i$ be the (not necessarily induced) subgraph of $G$ obtained by taking the union of $\overline{\tau}_0,\ldots,\overline{\tau}_i$.

\begin{lem}\label{face_trick}
  For any $G$ and $T_0$ defined as above, there exists an ordering $\mathcal{F}:=f_0,\ldots,f_{2n-3}$ of the faces of $G$ that defines a tripod decomposition $\mathcal{D}_\mathcal{F}:=\tau_0,\ldots,\tau_{2n-3}$ such that, for each $i\in\{0,\ldots,2n-3\}$,
  \begin{compactenum}[(i)]
    \item $G_i$ is biconnected;
    \item \label{three_faces} for each face $f$ of $G_i$, there exists $i_1,i_2,i_3\in\{0,\ldots,i\}$ such that
    \begin{compactenum}[(a)]
      \item $V(f)\subseteq \bigcup_{r=1}^3 \tau_{i_r}$; and
      \item $\tau_{i_r}\cap f$ is connected, for each $r\in\{1,2,3\}$.
    \end{compactenum}
  \end{compactenum}
\end{lem}

\citet{dujmovic.joret.ea:planar} prove \cref{face_trick} using top down strategy, that constructs the face sequence $\mathcal{F}$ iteratively.  To find the face $f_i$, they consider some face $f\not\in\{f_0,\ldots,f_{i-1}\}$ of $G_{i-1}$ and use (\ref{three_faces}) and Sperner's Lemma to show that there is an appropriate face $f_i$ of $G$ that is contained in $f$.  This proof strategy was used by \citet{morin:fast} to devise an $O(n\log n)$ time divide-and-conquer algorithm for computing a tripod decomposition satisfying the conditions of \cref{tripod_decomposition}.





\bibliographystyle{plainurlnat}
\bibliography{af2t}


\end{document}
