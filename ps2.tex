\documentclass{patmorin}
\listfiles
\usepackage{pat}
\usepackage{paralist}
\usepackage{dsfont}  % for \mathds{A}
\usepackage[utf8x]{inputenc}
\usepackage{skull}
\usepackage{paralist}
\usepackage{graphicx}
\usepackage[noend]{algorithmic}

\usepackage[normalem]{ulem}
\usepackage{cancel}
\usepackage{enumitem}

\usepackage{todonotes}

\usepackage[longnamesfirst,numbers,sort&compress]{natbib}

\newcommand{\Rho}{\mathrm{P}}

% \newcommand{\harpoon}{\overset{\rightharpoonup}}
\newcommand{\qa}{\overset{\rightharpoonup}{\varphi}}
\newcommand{\qb}{\overset{\rightharpoondown}{\varphi}}
\newcommand{\qap}{\overset{\rightharpoonup}{\sigma}}
\newcommand{\qbp}{\overset{\rightharpoondown}{\sigma}}

\usepackage[mathlines]{lineno}
\setlength{\linenumbersep}{2em}
% \linenumbers
% \rightlinenumbers
% \linenumbers
\newcommand*\patchAmsMathEnvironmentForLineno[1]{%
 \expandafter\let\csname old#1\expandafter\endcsname\csname #1\endcsname
 \expandafter\let\csname oldend#1\expandafter\endcsname\csname end#1\endcsname
 \renewenvironment{#1}%
    {\linenomath\csname old#1\endcsname}%
    {\csname oldend#1\endcsname\endlinenomath}}%
\newcommand*\patchBothAmsMathEnvironmentsForLineno[1]{%
 \patchAmsMathEnvironmentForLineno{#1}%
 \patchAmsMathEnvironmentForLineno{#1*}}%
\AtBeginDocument{%
\patchBothAmsMathEnvironmentsForLineno{equation}%
\patchBothAmsMathEnvironmentsForLineno{align}%
\patchBothAmsMathEnvironmentsForLineno{flalign}%
\patchBothAmsMathEnvironmentsForLineno{alignat}%
\patchBothAmsMathEnvironmentsForLineno{gather}%
\patchBothAmsMathEnvironmentsForLineno{multline}%
}


\newcommand{\coloured}[2]{{\color{#1}{#2}}}
\newenvironment{colourblock}[1]{\color{#1}}{}

\newcommand{\condref}[1]{(C\ref{#1})}

% Taken from
% https://tex.stackexchange.com/questions/42726/align-but-show-one-equation-number-at-the-end
\newcommand\numberthis{\addtocounter{equation}{1}\tag{\theequation}}


\setlength{\parskip}{1ex}


\DeclareMathOperator{\diam}{diam}
\DeclareMathOperator{\tw}{tw}
\DeclareMathOperator{\lca}{lca}

\DeclareMathOperator{\x}{x}
\DeclareMathOperator{\height}{height}
\DeclareMathOperator{\depth}{depth}
\DeclareMathOperator{\dist}{dist}
\DeclareMathOperator{\sh}{cbt}
\DeclareMathOperator{\cbt}{cbt}
\DeclareMathOperator{\sgn}{sgn}
\DeclareMathOperator{\dc}{dc}

\title{\MakeUppercase{An Optimal Algorithm for Product Structure in Planar Graphs}\thanks{This research was partly funded by NSERC.}}
\author{%
  Prosenjit Bose\thanks{School of Computer Science, Carleton University}\qquad
  Vida Dujmović\thanks{Department of Computer Science and Electrical Engineering, University of Ottawa}\qquad
  Pat Morin\footnotemark[1]\qquad
  Saeed Odak\footnotemark[2]}
    % }

\date{}


\newcommand{\colored}[2]{{\color{#1}#2}}

\usepackage{tabularx}


\begin{document}

% \begin{titlepage}
\maketitle

\begin{abstract}
  The Product Structure Theorem for planar graphs (Dujmović et al, 2019) states that any planar graph is contained in strong product of a planar $3$-tree, a path, and $3$-cycle.  We give a simple linear-time algorithm for finding this decomposition.
\end{abstract}
% \end{titlepage}

% \pagenumbering{roman}
% \tableofcontents
%
% \newpage
% \pagenumbering{arabic}

\section{Introduction}

For two graph $G$ and $X$, the notation $G\subseteq X$ denotes that $G$ is isomorphic to some subgraph of $X$.  Throughout this paper $P$ denotes the \emph{one way infinite path}, i.e., the path with vertex set $V(P):=\N$ and edge set $E(P):=\{\{i,i+1\}:i\in\N\}$.  The following \emph{planar product structure theorems} have recently been used as a key tool in resolving a number of longstanding open problems on planar graphs.

\begin{thm}[\citet{dujmovic.joret.ea:planar, ueckerdt.wood.ea:improved}]\label{meta}
  Let $P$ denote the infinite path.  For any planar graph $G$, there exists:
  \begin{compactenum}[(a)]
    \item \label{three_tree} a planar graph $H$ of treewidth at most $3$ and a path $P$ such that $G\subseteq H\boxtimes P\boxtimes K_3$ \cite{dujmovic.joret.ea:planar};
    \item a planar graph $H$ of treewidth at most $4$ and a path $P$ such that $G\subseteq H\boxtimes P\boxtimes K_2$; and
    \item a planar graph $H$ of treewidth at most $6$ and a path $P$ such that $G\subseteq H\boxtimes P$ \cite{ueckerdt.wood.ea:improved}.
  \end{compactenum}
\end{thm}

The proofs of these theorems are constructive and lead to $O(n^2)$ time algorithms.  \citet{morin:fast} showed that there exists an $O(n\log n)$ time algorithm to find the decomposition in \cref{meta}.\ref{three_tree}.  In the current note, we show that there exists a linear time algorithm for finding each of the three decompositions guaranteed by \cref{meta}.


\section{Preliminaries}

Throughout this paper we use standard graph theory terminology as used in the textbook by Diestel \cite{diestel:graph}.  All graphs discussed here are simple and finite.  For a graph $G$, $V(G)$ and $E(G)$ denote the vertex and edge sets of $G$, respectively.

\section{Quotient Graphs}

Given a graph $G$ and a partition $\mathcal{P}$ of $V(G)$, the \emph{quotient graph} $H:=G/\mathcal{Y}$ is the graph with vertex set $V(H):=\mathcal{P}$ and two nodes $X,Y\in V(H)$ are adjacent in $H$ if $G$ contains at least one edge $xy$ with $x\in X$ and $y\in Y$.

\subsection{Embeddings}

An \emph{embedding} $\psi$ of a graph $G$ associates each vertex $v$ of $G$ with a point $\psi(v)\in \R^2$ and each edge $vw$ of $G$ with a simple open curve $\psi(vw):(0,1)\to\R^2$ whose endpoints\footnote{The \emph{endpoints} of an open curve $\psi:(0,1)\to\R^2$ are the two points $\lim_{\epsilon\downarrow 0} \psi(\epsilon)$ and $\lim_{\epsilon\downarrow 0}\psi(1-\epsilon)$.} are $\psi(v)$ and $\psi(w)$.
We let $\psi(V(G)):=\{\psi(v):v\in V(G)\}$, $\psi(E(G)):=\bigcup_{vw\in E(G)} \psi(vw)$, and $\psi(G):=\psi(V(G))\cup\psi(E(G))$.  An embedding $\psi$ of $G$ is \emph{plane} if $\psi(vw)\cap\psi(V(G))=\emptyset$ and $\psi(vw)\cap\psi(xy)=\emptyset$ for each distinct pair of edges $vw,xy\in E(G)$.  A graph $G$ is \emph{planar} if it has an plane embedding. A \emph{triangulation} is an edge-maximal planar graph.

If $\psi$ is a plane embedding of a planar graph $G$, then we call the pair $(G,\psi)$ an \emph{embedded graph} and we will not distinguish between a vertex $v$ of $G$ and the point $\psi(v)$ or between an edge $vw$ of $G$ and the curve $\psi(vw)$.  Similarly, we will not distinguish between $G$ and the  point set $\psi(G)$.  Any cycle in an embedded graph defines a Jordan curve. For such a cycle $C$, $\R^2\setminus C$ has two components, one bounded and unbounded. We will refer to the bounded component as the \emph{interior} of $C$ and the unbounded component as the \emph{exterior} of $C$.  If $G$ is an embedded triangulation, then the subgraph of $G$ consisting of all edges and vertices of $G$ contained in the closure of the interior of $C$ is called a \emph{near-triangulation}.

Each component of $\R^2\setminus G$ is a \emph{face} of $G$ and we let $F(G)$ denote the set of faces of $G$.  If $G$ is $2$-connected then, for any face $f\in F(G)$, the set of vertices and edges of $G$ contained in the boundary of $f$ form a cycle.  We may therefore treat a face $f$ of a $2$-connected graph $G$ as a component of $\R^2\setminus G$ or as the cycle of $G$ on the boundary of $f$, relying on context to distinguish between the two usages.  Note that every embedded graph contains exactly one face, the \emph{outer face} that is unbounded.

\subsection{Duals and Cotrees}

The \emph{dual} $G^\star$ of an embedded planar graph $G$ is the graph with vertex set $V(G^\star):=F(G)$ and edge set $E(G^{\star}):=\{fg\in \binom{F(G)}{2}:E(f)\cap E(g)\neq\emptyset\}$.  If $T$ is a spanning tree of an embedded triangulation $G$ then the \emph{cotree} $\overline{T}$ of $(G,T)$ is the graph $\overline{T}:=G^\star-\{ab\in E(G^*):E(a)\cap E(b)\setminus E(T)\neq\emptyset\}$.  It is well known that such a cotree is a spanning tree of $G^\star$.

\subsection{Paths and Distances}

A \emph{path} in $G$ is a (possibly empty) sequence of vertices $v_0,\ldots,v_r$ with the property that $v_{i-1}v_i\in E(G)$, for each $i\in\{1,\ldots,r\}$.  The \emph{endpoints} of a path $v_0,\ldots,v_r$ are the vertices $v_0$ and $v_r$.
% We will treat a path $v_0,\ldots,v_r$ interchangeably with the subgraph of $G$ having vertex set $\{v_0,\ldots,v_r\}$ and edge set $\{v_{i-1}v_i:i\in\{1,\ldots,r\}\}$.
The \emph{length} of a non-empty path $v_0,\ldots,v_r$ is the number, $r$, of edges in the path.  For two vertices $v$ and $w$ in a connected graph $G$, $\dist_G(v,w)$ denotes the minimum length of a path in $G$ that contains both $v$ and $w$.
% For a non-empty subset $S\subseteq V(G)$, $\dist_G(v,S):=\min\{\dist_G(v,w):w\in S\}$.
%

\subsection{Trees, Depth, Ancestors, and Descendants}

Let $T$ be a tree rooted at a vertex $v\in V(T)$.  For any vertex $w\in V(T)$, $P_T(w)$ denotes the path in $T$ from $v$ to $w$.  For any $w_0\in V(T)$, any prefix $w_0,\ldots,w_r$ of $P_T(w_0)$ is called an \emph{upward path} in $T$; $v_0$ is the \emph{lower endpoint} of this path and $v_r$ is the \emph{upper endpoint}.

The \emph{$T$-depth} of a node $w\in V(T)$ is $\depth_T(w):=\dist_T(v,w)$.  A vertex $a\in V(T)$ is an \emph{ancestor} of $v\in V(T)$ if $a\in V(P_T(v))$.


\subsection{Lowest Common Ancestors}

For any two vertices $v,w\in V(T)$, the \emph{lowest common ancestor} $\lca_T(v,w)$ of $v$ and $w$ is the vertex $a$ of $P_T(v)\cap P_T(w)$ that maximizes $\depth_T(a)$.  The \emph{lowest commmon ancestor problem} is a well-studied data structuring problem that asks to preprocess a given $n$-vertex rooted tree so that one can quickly return $\lca_T(v,w)$ for any two nodes $v,w\in V(T)$. A number of optimal solutions to this problem exist that, after $O(n)$ time preprocessing using $O(n)$ space, can answer queries in $O(1)$ time \cite{X}.  Recent work in this area includes simple and practical data structures that achieve this optimal performance \cite{X}.

% \subsection{Nearest Marked Ancestors}
%
% The \emph{nearest marked ancestor problem} is another well-studied data structuring problem that asks to preprocess an $n$-vertex rooted tree $T$ whose nodes are all initially \emph{unmarked} with the exception of the root which is \emph{marked}.  The data structure supports two operations:
% \begin{inparaenum}[(i)]
%     \item \emph{mark} a node $u$ of $T$;
%     \item find the \emph{nearest marked ancestor} of a node $u$ of $T$, i.e., the ancestor of $u$ of maximum depth that is marked.
% \end{inparaenum}
% There exists data structures for this problem that use $O(n)$ space and preprocessing and can perform each of these operations in $O(1)$ time  \cite{X}.  Especially simple and efficient solutions exist when the operations are restricted so that the set of marked nodes always induce a connected subtree of $T$.  In this paper we only require this simple version of the data structure.

\section{Tripod Decompositions}

Let $G$ be a $n$-vertex triangulation and $T$ be a spanning tree of $G$. For a face $uvw$ of $G$, a \emph{$(G,T)$-tripod} with \emph{crotch} $uvw$ is the vertex set of three disjoint (and each possibly empty) vertical paths whose lower endpoints are $u$, $v$, and $w$.  A \emph{$(G,T)$-tripod decomposition} is a partition of $V(G)$ into $(G,T)$-tripods.  \citet{dujmovic.joret.ea:planar} proved the following result:

\begin{thm}\label{tripod_decomposition}
  Let $G$ be a triangulation and $T$ be a spanning tree of $G$.  Then there exists a $(G,T)$-tripod decomposition $\mathcal{Y}$ such that $G/\mathcal{Y}$ has treewidth at most $3$.
\end{thm}

It is straightforward to verify that \cref{tripod_decomposition} implies \cref{meta}.\ref{three_tree} by first triangulating the given graph and then taking $T$ to be a breadth-first search tree of the resulting triangulated graph \cite[Observation~35]{dujmovic.joret.ea:planar}.

% \begin{remark}
%   The result of \citett{dujmovic.joret.ea:planar} provides tripods with somewhat more structure.  Every tripod has three legs whose lower endpoints are on a common face of $G$.  It is possible to fix, in advance, an $S$-rooted BFS tree $T$ and produce tripods whose legs are paths in $T$.
% \end{remark}

\subsection{Tripod Decompositions from Face Orderings}
\label{orderings}

We now describe how $(G,T)$-tripod decompositions can be obtained from total orders on the faces of $G$.  For any subgraph $f$ of $G$, we define $Y_T(f):=\bigcup_{v\in V(F)} P_T(v)$.  Let $\mathcal{F}:=f_0,\ldots,f_{2n-3}$ be a total ordering of the faces of $G$. Let $G_{-1}$ denote the graph with no vertices and, for each $i\in\{0,\ldots,2n-1\}$, define the graph $G_i:=\bigcup_{j=0}^i (f_j\cup Y_F(f_j))$ and let $Y_i:=V(G_i)\setminus V(G_{i-1})$.
Let $\mathcal{G_F}:=\langle G_0,\ldots,G_{2n-3}\rangle$ and
observe that $\mathcal{Y_F}:=\{Y_0,\ldots,Y_{2n-3}\}$ is a tripod decomposition of $G$.


Let us say that a sequence $\mathcal{F}:=f_0,\ldots,f_i$ of distinct faces of $G$ is \emph{good} if the resulting sequence of graphs $\mathcal{G}_\mathcal{F}:=G_0,\ldots,G_i$ and tripods $\mathcal{Y}_\mathcal{F}:=Y_0,\ldots,Y_i$ satisfy the following condition:  For each $j\in\{0,\ldots, i\}$ and each face $f$ of $G_j$,
\[
   |\{\ell\in\{0,\ldots,j\}: V(f)\cap Y_{\ell}\neq\emptyset\}|\le 3 \enspace .
\]
In words, each face of $G_j$ has vertices from at most three tripods of $Y_0,\ldots,Y_j$ on its boundary.  \citet{dujmovic.joret.ea:planar} prove \cref{tripod_decomposition} by proving the next lemma.

% In words, this lemma states that if each face of each graph in $\mathcal{G}_{\mathcal{F}}$ has vertices from at most three tripods on its boundary then $G/\mathcal{Y}_\mathcal{F}$ has treewidth at most $3$.

\begin{lem}\label{face_trick}
  Let $G$ be a triangulation $G$ with a vertex $v_0$ on its outer face $f_0$ and let $T$ be a spanning tree of $G$ rooted at $v_0$.  Then there exists a good sequence $\mathcal{F}:=f_0,\ldots,f_{2n-3}$ containing all the faces of $G$.
\end{lem}

\begin{rem}
  \cref{face_trick} is stated in terms of a total order on $F(G)$ only for convenience.  Indeed, consider the partial order $\prec$ defined as follows:  For each $i\in\{1,\ldots,2n-3\}$ and each $j\in I_{f_i}$, $f_j\prec f_i$.  It is straightforward to check that any linearization of this total order will result in the same tripod decomposition $\{Y_0,\ldots,Y_{2n-3}\}$.
\end{rem}

\citet{dujmovic.joret.ea:planar} prove \cref{face_trick} by giving a recursive algorithm that constructs the face sequence $\mathcal{F}$.  For a face $f$ of $G_j$, define the set $I_f:=\{\ell\in\{0,\ldots,j\}:V(f)\cap Y_\ell\neq\emptyset\}$.  They begin with the outer face $f_0$ of $G$.  To find the face $f_i$, $i>0$, they consider some face $f\not\in\{f_0,\ldots,f_{i-1}\}$ of $G_{i-1}$ and use Sperner's Lemma to show that there is an appropriate face $f_i$ (called a \emph{Sperner triangle}) of $G$ that is contained in $f$. In particular, $f_i$ is chosen so that the three vertical paths in $Y_F(f_i)$ lead back to each of the (at most 3) tripods in $\{Y_j:j\in I_f\}$. See \cref{sperner}

\begin{figure}
  \begin{center}
    \begin{tabular}{ccc}
      \includegraphics{figs/sperner-1} &
      \includegraphics{figs/sperner-2} &
      \includegraphics{figs/sperner-3} \\
      (a) & (b) & (c)
    \end{tabular}
  \end{center}
  \caption{Each face $f$ in $G_{i-1}$ is bounded by three tripods $Y_{a_f}$, $Y_{b_f}$, and $Y_{c_f}$ and the tripod $Y_i$ is chosen so that it connects each of these.}
  \label{sperner}
\end{figure}

This immediately leads to a divide-and-conquer algorithm by recursing on each of the at most three new faces in $S_i:=F(G_i)\setminus F(G_{i-1})\setminus \{f_i\}$.  The Sperner triangle $f_i$ can easily be found in time proportional to the number of faces of $G$ in the interior of $f$.  However, because the resulting recursion is not necessarily balanced, a straightforward implementation of this yields an algorithm with $\Theta(n^2)$ worst-case running time.

\citet{morin:fast} later showed that, using an appropriate data structure for $T$, this approach can be implemented in such a way that the resulting algorithm runs in $O(n\log n)$ time.  Essentially, Morin's algorithm works by finding the Sperner triangle $f_i$ in time proportional to the minimum number of faces of $G$ contained in any of the faces in $S_i$.  In the next section, we will show that, by using an appropriate data structure for the cotree $\overline{T}$, the Sperner triangle $f_i$ can be found in constant time, yielding an $O(n)$ time algorithm.

By now, our presentation of this material differs somewhat from that in \cite{dujmovic.joret.ea:planar,ueckerdt.wood.ea:improved}.  Therefore, we now pause to explain how \cref{face_trick} implies \cref{tripod_decomposition}.  To do this, we show that there exists a chordal graph $H$ whose largest clique has size at most $4$ and that contains $G/\mathcal{Y_F}$. For each $i\in\{0,\ldots,2n-3\}$ and each face $f$ of $G_i$, the graph $H$ contains a clique on $\{Y_j:j\in I_f\}$.  To do this, for each $i\in\{1,\ldots,2n-3\}$ we let $f$ be the face of $G_{i-1}$ that contains $f_i$ and we form a clique on $\{Y_i\}\cup\{Y_j:j\in I_f\}$.  Inductively, the elements of $\{Y_j:j\in I_f\}$ already form a clique, so this operation is equivalent to attaching $Y_i$ to all the vertices of an existing clique of size at most $3$. Therefore, this results in a chordal graph $H$ whose largest clique has size at most $4$ and therefore $H$ has treewidth at most $3$ \cite{gavril:intersection}.

\section{An $O(n)$-Time Algorithm}

Refer to \cref{baby_sperner_fig} for an illustration of the the following (probably well-known) baby version of Sperner's Lemma:

\begin{figure}
  \begin{center}
    \includegraphics{figs/baby_sperner}
  \end{center}
  \caption{\cref{baby_sperner}}
  \label{baby_sperner_fig}
\end{figure}

\begin{lem}\label{baby_sperner}
  Let $N$ be a near-triangulation with outer face $v_0,\ldots,v_r$ and colour each vertex of $N$ red or blue in such a way that $v_0,\ldots,v_i$ are coloured red for some $i\in\{0,\ldots,r-1\}$ and $v_{i+1},\ldots,v_r$ are coloured blue.  Then there exists a path $w_0,\ldots,w_k$ in $N^\star$ such that
  \begin{compactenum}
    \item $w_0$ is the inner face of $N$ with $v_0v_r$ on its boundary;
    \item $w_k$ is the inner face of $N$ with $v_iv_{i+1}$ on its boundary; and
    \item for each $i\in\{1,\ldots,r\}$, the single edge in $E(w_{i-1})\cap E(w_i)$ has an endpoint of each colour.
  \end{compactenum}
\end{lem}

\begin{proof}
  If $w_0=w_k$, the lemma is immediately true, so assume $w_0\neq w_k$.
  Say that an edge of $n$ is \emph{bichromatic} if one of its endpoints is red and the other is blue.  Any edge that is not bichromatic is \emph{monochromatic}.  The outer face $f_0$ of $N$ has exactly two bichromatic edges $v_0v_r$ and $v_iv_{i+1}$ and any inner face of $N$ has either zero or two bichromatic edges.  Consider the subgraph $H$ of $N^\star$ obtained removing each edge $fg\in E(N^\star)$ such that the edge in $E(f)\cap E(g)$ is monochromatic.  Every vertex in $H$ has degree $0$ or $2$, so each connected component of $H$ is either an isolated vertex of a cycle.  The face $f_0$ has degree $2$ so it is contained in a cycle $C$ of $H$.  The two neighbours of $f_0$ in $H$ are $w_0$ and $w_k$. Therefore $C$ contains a path $w_0,\ldots,w_k$ that satisfies the conditions of the lemma.
\end{proof}

The next lemma, which is the main observation in this paper, allows us to use lowest common ancestor data structures to find Sperner triangles in constant time.

\begin{lem}\label{lca_sperner}
  Let $G$ be a triangulation $G$ with a vertex $v_0$ on its outer face $f_0$; let $T$ be a spanning tree of $G$ rooted at $v_0$; let $\overline{T}$ be the cotree of $(G,T)$ rooted at $f_0$; let $f_0,\ldots,f_{i-1}$ be a good sequence of faces of $G$ that yields a sequence $\mathcal{G_F}:=G_0,\ldots,G_{i-1}$ of graphs and a sequence $\mathcal{Y_F}:=Y_0,\ldots,Y_{i-1}$ of tripods; let $f\not\in \{f_0,\ldots,f_{i-1}\}$ be a face of $G_{i-1}$ with $|I_f:=\{j\in\{0,\ldots,i-1\}:V(I_f)\cap Y_j\neq\emptyset\}|=3$, and let $L\subseteq F(G)$ contain every face $g\in F(G)$ such that
  \begin{compactenum}[(i)]
      \item $g$ is contained in the interior of $f$;
      \item $g$ contains an edge $vw\in E(f)$ with $v\in Y_a$ and $w\in Y_b$ for some distinct $a,b\in\{0,\ldots,i-1\}$.
  \end{compactenum}
  Then there exists a face $f_i$ of $G$ that is contained in $f$ and such that
  \begin{compactenum}[(a)]
    \item $f_0,\ldots,f_i$ is good; and
    \item $f_i :=\lca_{\overline{T}}(x,y)$ for some $x,y\in L$.
  \end{compactenum}
\end{lem}

\begin{proof}
  Let $N$ be the near-triangulation consisting of all vertices and edges of $G$ in the closure of the interior of $f$ and let $\{a,b,c\}:=I_f$.  Colour each vertex $v$ of $N$ with a colour in $\{a,b,c\}$ depending on whether the first vertex of $P_{T}(v)$ contained in $V(f)$ belongs to $Y_a$, $Y_b$, or $Y_c$, respectively.  Say that an edge or face of $N$ is \emph{monochromatic}, \emph{bichromatic}, or \emph{trichromatic} if it contains vertices of one, two, or three colours, respectively.  It is straightforward to verify that any trichromatic inner face $f_i$ of $N$ will satisfy Condition~(a). By Sperner's Lemma, such a trichromatic face $f_i$ does indeed exist. All that is needed is to check that $f_i$ satisfies Condition~(b).  There are two cases to consider:
  \begin{enumerate}
    \item $f_i$ contains an edge $uv\in E(f)$.  Since $f_i$ is trichromatic,  $u$ and $v$ have different colours, so $f_i\in L$.  Furthermore, $f_i=\lca_{\overline{T}}(f_i,f_i)$, so $f_i$ satisfies Condition~(b).

    \item $f_i$ contains no edge of $f$.  See \cref{lca_proof}.  The graph $G_i$ contains exactly four faces $g_1',g_2',g_3',f_i$ that are not faces of $G_{i-1}$, but that are contained in $f$. For each $j\in\{1,2,3\}$, the face $g_j'$ contains exactly one face $g_i$ of $G$ such that $E(g_j)\cap E(f)$ contains a bichromatic edge in $E(f)$.

    \begin{figure}
      \begin{center}
        \begin{tabular}{cc}
          \includegraphics{figs/lca_proof-1} &
          \includegraphics{figs/lca_proof-2} \\
          $G_i$ & $G$
        \end{tabular}
      \end{center}
      \caption{The cotree $\overline{T}$ contains disjoint paths from $f_i$ to each of $g_1$, $g_2$, and $g_3$.}
      \label{lca_proof}
    \end{figure}

    We claim that, for each $j\in\{1,2,3\}$, $\overline{T}$ contains a path from $f_i$ to $g_j$ and, with the exception of $f_i$, these paths are disjoint.  In order to show this, we first argue that no bichromatic edge in the interior of $f$ is an edge of $T$.  Consider any $uv\in E(N)\setminus E(f)$ where $u$ is the $T$-parent of $v$.  If $v\not\in V(f)$ then, by definition, $v$ has the same colour as $u$. The case where $v\in V(f)$ and $u\not\in V(f)$ can not occur since $v\in V(f)$ implies that $P_T(v)\subseteq G_{i-1}$, but $u\not\in V(G_{i-1})$.  Similarly, the case in which $u\in V(f)$ and $v\in V(f)$ can not occur since this implies that $P_T(v)\subseteq G_{i-1}$, but $uv\not\in E(G_{i-1})$.

    Now, for each $j\in\{1,2,3\}$, let $N_j$ be the near-triangulation consisting of the faces and edges of $G$ in the interior and on the boundary of $g_j'$.  Then the colouring of $N$, restricted to $N_j$ satisfies the conditions of \cref{baby_sperner} so $N^\star$ contains a path $w_{-1},w_0,\ldots,w_{k}$ with $w_{-1}=f_i$, $w_{k}=g_j$, and $w_0,\ldots,w_{k-1}$ contained in $g_j'$ and such that, for each $\ell\in\{0,\ldots,k\}$, the edge shared by $w_{\ell-1}$ and $w_\ell$ is bichromatic.  Therefore $w_{-1},w_0\ldots,w_k$ is a path in $\overline{T}$ from $f_i$ to $g_j$.

    Each of the three paths from $f_i$ to each of $g_1,g_2,g_3$ are disjoint except for their shared starting point $f_i$.  Therefore, since $f_i$ has only one $\overline{T}$-parent, at least two of $g_1,g_2,g_3$ are $\overline{T}$-descendants of $f_i$, say $g_1$ and $g_2$.  Therefore $f_i:=\lca_{\overline{T}}(g_1,g_2)$, as required. \qedhere
  \end{enumerate}
\end{proof}

\cref{lca_sperner} implies that, given the triangles $q_1$, $q_2$ and $q_3$, one can use a lowest common ancestor data structure to find, in constant time a constant-size set $S:=\{\lca_{\overline{T}}(g_\alpha,g_\beta):\alpha,\beta\in\{1,2,3\}\}$ of candidate faces, one of which is the Sperner triangle $f_i$.
This quickly leads to an $O(n)$ time algorithm that, like the algorithm of \citet{morin:fast} uses a nearest marked ancestor data structure on $T$ to recognize which of these candidate faces is the correct choice for $f_i$.

We now show that the nearest marked ancestor data structure can be eliminated entirely. Instead, we can identify the face $f_i\in S$ using only lowest common ancestor queries.  Using the notation from \cref{lca_sperner}, let $C$ be the outer face of $N$ and let $g_1$, $g_2$, and $g_3$ be the inner faces of $N$ having bichromatic edges of $C$ on their boundary where $g_1$ has an edge of $C$ coloured with $a$ and $b$, $g_2$ has an edge of $C$ coloured with $b$ and $c$, and $g_3$ has an edge of $C$ coloured with $a$ and $c$.  There are three cases, illustrated in \cref{lca_cases}.

\begin{figure}
  \begin{center}
    \begin{tabular}{ccc}
      \includegraphics{figs/lca_cases-1} &
      \includegraphics{figs/lca_cases-2} &
      \includegraphics{figs/lca_cases-3} \\
      (1) & (2) & (3)
    \end{tabular}
  \end{center}
  \caption{Identifying a Sperner triangle $f_i$ using lowest common ancestor queries over $g_1$, $g_2$, and $g_3$.}
  \label{lca_cases}
\end{figure}

\begin{enumerate}
  \item If $g_\alpha=g_\beta$ for some distinct $\alpha,\beta\in\{1,2,3\}$  then $g_\alpha$ is trichromatic and we can take $f_i:=g_\alpha$.  In the proof of \cref{lca_sperner}, this corresponds to the case where $f_i$ has an edge of $C$ on its boundary.

  % \item If $\lca_{\mathcal{T}}(g_\alpha,g_\beta)=g_\gamma$ for distinct $\alpha,\beta,\gamma$ then $g_\beta$ is trichromatic and we are done.

  \item If $\lca_{\mathcal{T}}(g_\alpha,g_\beta)=g_\beta$ and $\lca_{\overline{T}}(g_\beta,g_\gamma)=q$ for distinct  $\alpha$, $\beta$, $\gamma$, and $q\ne g_\beta$ then $g_\beta$ is trichromatic and we are done.  Note that this includes the case where $g_\gamma=q$ or $g_\beta=q$.  In the proof of \cref{lca_sperner} this also corresponds to the case where $f_i$ has an edge of $C$ on its boundary.

  \item If $\lca_{T}(g_\alpha,g_\beta)=p$, $\lca_{T}(g_\beta,g_\gamma)=q$,  $\lca_{\overline{T}}(g_\alpha,g_\gamma)=q$, and $\lca_{\overline{T}}(p,q)=q$ for distinct $\alpha$, $\beta$, $\gamma$, $p$, and $q$, then $p$ is trichromatic and we are done. This includes the possibility that  $q=g_\gamma$.  In the proof of \cref{lca_sperner} this corresponds to the case where $f_i$ has no edge of $C$ on its boundary.
\end{enumerate}
The preceding cases are exhaustive since, after ruling out Case~1, the minimal subtree of $\overline{T}$ that contains $g_1$, $g_2$, and $g_3$ is a binary tree whose leaves are a subset of $\{g_1,g_2,g_3\}$.  Any binary tree with $n\ge 3$ nodes and at most $3$ leaves has either $2$ leaves (Case~2) or $3$ leaves (Case 3).

\begin{rem}\label{colour_remark}
  The process described above not only makes it possible to identify $f_i$ in constant time, but also makes it possible to determine the colours of each of the three vertices of $f_i$ in constant time.
\end{rem}



\begin{thm}
  There exists an $O(n)$ time algorithm that, given any $n$-vertex triangulation $G$ and any spanning tree $T$ of $G$, produces a $(G,T)$-tripod decomposition $\mathcal{Y}$ such that $\tw(G/\mathcal{T})\le 3$.
\end{thm}

\begin{proof}
  Let $f_0$ be any face of $G$ incident to $v_0$.  First we compute the cotree $\overline{T}$ of $(G,T)$ and construct  a lowest common ancestor data structure for $\overline{T}$ in $O(n)$ time that allows us to compute $\lca_{\overline{T}}(f,g)$ for any two faces $f,g\in F(G)$ in $O(1)$ time.

  We construct the good sequence $f_0,\ldots,f_{2n-3}$ recursively. Conceptually, during any recursive invocation, the input is a near-triangulation $N$ bounded by a cycle $C$ in $G$ whose vertices belong to at most three tripods computed in previous steps.  Each vertex of $G$ starts initially \emph{unmarked} and we \emph{mark} a vertex once we have placed it in a tripod.  The precise input to a recursive invocation is defined as follows:
  \begin{enumerate}
    \item If $C$ has three tripods of on its boundary then the input consists of the three inner faces $g_1$, $g_2$, and $g_3$ of $N$ that contain bichromatic edges of $C$. In this case, the discussion above allows us to find the Sperner triangle $f_i$ in $O(1)$ time using a constant number of queries in the lowest common ancestor data structure for $\overline{T}$.

    \item If $C$ has only two tripods on its boundary, then the input consists of two inner faces $g_1$, $g_2$, of $N$ with bichromatic edges of $C$ on their boundary.  In this case, we let $f_i:=g_1$ or $f_i=g_2$, either choice satisfies our requirements.

    \item If $C$ has only one tripod on its boundary then the input consists of any inner face of $N$ with that contains an edge of $C$.
\end{enumerate}
Once we have found the Sperner triangle $f_i$, we can compute the tripod $Y_i$ by following the path in $T$ from each vertex of $f_i$ to its nearest marked ancestor in $T$.  This takes $O(1+|Y_i|)$ time.  Once we have done this, we have also found the at most three bichromatic edges of $G_i$ that are needed to perform the at most three recursive invocations on the near triangulations whose outer faces are each of the new faces in $F(G_i)\setminus F(G_{i-1})\setminus\{f_i\}$.

Each recursive invocation adds a new face $f_i$ to the good face sequence $f_0,\ldots,f_{2n-3}$ and takes $O(1+|Y_i|)$ time.  Since $Y_0,\ldots,Y_{2n-3}$ is a partition of $V(G)$, the running time of this algorithm is therefore $\sum_{i=0}^{2n-3} O(1+|Y_i|) = O(n)$.
\end{proof}


\section{Variations}

In this section we show that the following two variations on the product structure theorem also have linear-time algorithms:
\begin{itemize}
  \item For every planar graph $G$, there exists a planar graph $H$ of  treewidth at most $6$ and a path $P$ such that $G\subseteq H\boxtimes P$ \cite{ueckerdt.wood.ea:improved}.
  \item For every planar graph $G$, there exists a planar graph $H$ of treewidth at most $4$ and a path $P$ such that $G\subseteq H\boxtimes P\boxtimes K_2$.
\end{itemize}

Both of these results are obtained by recursive algorithm similar to that described for computing tripod decompositions.

\subsection{Bipod Decompositions}

We begin with the second decomposition, which is simpler, but seems not to have been noted before.  We select a sequence $\mathcal{E}:=e_1,\ldots,e_k$ of distinct edges of $G$, which define a sequence of graphs $\mathcal{G_E}:=G_{0},\ldots,G_k$ where $G_i:=\bigcup_{j=0}^i P_T(e_j)$ and a sequence of \emph{bipods} $\mathcal{I_E}:=I_0,\ldots,I_k$ where $I_i=V(G_i-G_{i-1})$.  We call $\mathcal{E}$ \emph{good} if, for each $i\in\{0,\ldots,k\}$ and each face $f\in F(G_i)$, $V(f)$ has a non-empty intersection with at most $4$ bipods in $I_0,\ldots,I_i$.  Exactly the same argument used in \cref{orderings} to show that $G/\mathcal{Y_F}$ is contained in a chordal graph of maximum clique size $4$ shows that if $\mathcal{E}$ is a good edge sequence that produces a partition $\mathcal{I_E}$ of $V(G)$, then $G/\mathcal{I_E}$ is contained in a chordal graph of maximum clique size $5$, so $G/\mathcal{I_E}$ has treewidth at most $4$.

\begin{figure}
  \begin{center}
    \begin{tabular}{cc}
      \includegraphics{figs/bipods-1} &
      \includegraphics{figs/bipods-2}
    \end{tabular}
  \end{center}
  \caption{Choosing the next $e_i$ in a  good edge sequence.}
\end{figure}

The edge $e_0$ is any edge of $E(f_0)\setminus E(T)$.  To choose the edge $e_i$ we consider any face $f\in F(G_{i-1})\setminus F(G)$.  This face $f$ has at most four bipods $I_a$, $I_b$, $I_c$, and $I_{c'}$ on its boundary.  As before we colour vertices in near triangulation $N$ bounded by $f$ with colours $a$, $b$, and $c$; using colour $c$ for vertices that lead to $c$ or to $c'$.  Again, Sperner's lemma implies that that $N$ contains an inner face $f_i$ that is trichromatic.  In the graph $G_i^+:=G_{i-1}\cup P_T(f_i)$, each newly introduced face in $F(G_{i}^+)\setminus F(G_{-1})\setminus\{f_i\}$ is incident on at most two of $I_a$, $I_b$ or $I_{c}\cup I_{c'}$.  Two of these faces are incident at most two of $I_a$, $I_b$, $I_c$ or $I_c'$.  These two faces are separated by one of the legs of $f_i$ whose lower endpoint is a vertex $v_i\in V(f_i)$.  We take $e_i$ to be the unique edge in $f_i-v_i$.  We can easily check that this results in a good sequence $e_0,\ldots,e_i$ of edges.  This process terminates when $G_k=G$.  That is, each edge of $G$ is contained in $P_T(e_i)$ for some $i\in\{0,\ldots,k\}$.

Algorithmically, this process requires almost no changes from the process used to find good face sequences.  Each recursive subproblem is defined by the at most $4$ inner faces of a near-triangulation that contain a edge of the outer face that spans two bipods.  The triangle $f_i$ is identified by using lowest common ancestor queries on three of these faces.  Using \cref{colour_remark}, the vertex $v_i$ of $f_i$ that is not part of $e_i$ can also be identified in constant time.  Colouring the vertices of $I_i$ and finding the at most four new faces of $G$ required to define the at most two new recursive subproblems can be done in $O(1+|I_i|)$ time.  This establishes the following result:

\begin{thm}
  There exists an $O(n)$ time algorithm that, given any $n$-vertex triangulation $G$ and any spanning tree $T$ of $G$, produces a $(G,T)$-bipod decomposition $\mathcal{I}$ such that $\tw(G/\mathcal{T})\le 3$.
\end{thm}


\subsection{Monopod Decompositions}

Finally the decomposition given by \citet{ueckerdt.wood.ea:improved} works similarly, but each part of the partition is a single upward path in $T$.  Each recursive subproblem is a near triangulation $N$ whose outer face intersects at most $5$ parts of the partition. This gives up to five bichromatic edges on the outer face of $N$, and up to five inner faces of $N$ that each contain at least one of these edges.


\todo[inline]{Continue on Monday}



\section*{Acknowledgement}

This research was initiated at the BIRS 21w5235 Workshop BIRS on Graph Product Structure Theory, held November 21--26, 2021 at the Banff International Research Station.  The authors are grateful to the workshop organizers and participants for providing a stimulating research environment.


\bibliographystyle{plainurlnat}
\bibliography{ps2}


\end{document}
